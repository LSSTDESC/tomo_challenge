\PassOptionsToPackage{usenames,dvipsnames}{xcolor}
\documentclass[twocolumn,twocolappendix]{aastex63}

% Add your own macros here:
\pdfoutput=1 %for arXiv submission
%\usepackage{amsmath,amssymb,amstext}
\usepackage{amsmath,amstext}
\usepackage[T1]{fontenc}
\usepackage{apjfonts}
\usepackage{ae,aecompl}
\usepackage[utf8]{inputenc}
\usepackage[figure,figure*]{hypcap}

\usepackage{url}
\urlstyle{same}

\usepackage{lineno}
\linenumbers
%\modulolinenumbers[2]

\newcommand{\placeholder}[1]{\textit{PLACEHOLDER: #1}}


% header settings
\shorttitle{Tomographic binning optimization}
\shortauthors{Author A et al.\ (LSST~DESC)}

% ======================================================================

\begin{document}
\title{The LSST DESC 3x2pt Tomography Optimization Challenge}
\collaboration{1000}{The LSST Dark Energy Science Collaboration (LSST DESC)}
\noaffiliation

% paper leads

\author[0000-0001-9789-9646]{Joe Zuntz}
\affiliation{Institute for Astronomy, University of Edinburgh, Edinburgh EH9 3HJ, United Kingdom}

\author[0000-0001-7956-0542]{Fran\c{c}ois Lanusse}
\affiliation{AIM, CEA, CNRS, Universit\'e Paris-Saclay, Universit\'e Paris Diderot, Sorbonne Paris Cit\'e, F-91191 Gif-sur-Yvette, France}

%main partticipants and text contributors  (alphabetical)
\author{Bela Abolfathi}
\noaffiliation

\author[0000-0002-4598-9719]{David Alonso}
\affiliation{Astrophysics, University of Oxford, DWB, Keble Road, Oxford OX1 3RH, UK}


\author{Abby Bault}
\noaffiliation

\author[0000-0003-4383-2969]{Cl\'{e}cio R. Bom}
\affiliation{Centro Brasileiro de Pesquisas F\'isicas, Rua Dr. Xavier Sigaud 150, 22290-180 Rio de Janeiro, RJ, Brazil}
\affiliation{Centro Federal de Educa\c{c}\~{a}o Tecnol\'{o}gica Celso Suckow da Fonseca, Rodovia M\'{a}rcio Covas, lote J2, quadra J - Itagua\'{i} (Brazil)}


\author{Massimo Brescia}
\affiliation{INAF - Osservatorio Astronomico di Capodimonte, via Moiariello 16, I-80131, Napoli Italy}

\author{Adam Broussard}
\noaffiliation

\author[0000-0002-1590-6927]{Jean-Eric Campagne}
\affiliation{Universit\'e Paris-Saclay, CNRS/IN2P3, IJCLab, 91405 Orsay, France}

\author[0000-0002-3787-4196]{Stefano Cavuoti}
\affiliation{INAF - Osservatorio Astronomico di Capodimonte, via Moiariello 16, I-80131, Napoli Italy}
\affiliation{INFN - Sezione di Napoli, via Cinthia 21, I-80126 Napoli, Italy}
\affiliation{Department of Physics ``E. Pancini'', University of Naples Federico II, via Cintia, 21, I-80126 Napoli, Italy}


\author{Eduardo S. Cypriano} $^3$
\affiliation{Universidade de S\~ao Paulo, IAG, Rua do Mat\~ao 1225, S\~ao Paulo, SP, Brazil}


\author{Bernardo M. O. Fraga}
\affiliation{Centro Brasileiro de Pesquisas F\'isicas, Rua Dr. Xavier Sigaud 150, 22290-180 Rio de Janeiro, RJ, Brazil}

\author{Eric Gawiser}
\noaffiliation


\author[0000-0002-0226-9893]{Elizabeth J. Gonzalez} $^{4,5}$.
\affiliation{Instituto de Astronom\'{\i}a Te\'orica y Experimental (IATE-CONICET), Laprida 854, X5000BGR, C\'ordoba, Argentina.}

\affiliation{Observatorio Astron\'omico de C\'ordoba, Universidad Nacional de C\'ordoba, Laprida 854, X5000BGR, C\'ordoba, Argentina.}


\author{Dylan Green}
\noaffiliation

\author{Peter Hatfield}
\noaffiliation

\author{David Kirkby}
\noaffiliation


\author{Alex Malz}
\noaffiliation

\author{Andrina Nicola}
\noaffiliation

\author{Erfan Nourbakhsh}
\noaffiliation

\author{Andy Park}
\noaffiliation

\author{Anze Slosar}
\noaffiliation

\author{Gabriel Teixeira}
\affiliation{Centro Brasileiro de Pesquisas F\'isicas, Rua Dr. Xavier Sigaud 150, 22290-180 Rio de Janeiro, RJ, Brazil}

\author{Angus Wright}
\noaffiliation

% minor contributions and/or text writers (alphabetical)


\begin{abstract}
In this paper we present the results of the DESC 3x2pt tomography challenge, which aims to compare strategies for optimizing the photometric binning required for the main LSST 3x2pt analysis. This task is made particularly delicate in the context of a metacalibrated lensing survey, as only the photometry from the bands  included in the metacalibration process (riz and potentially g) can be used to define this tomographic binning. Using a set of realistic simulations including true galaxy redshifts and simulated photometry, the goal of the challenge is to propose a bin assignment strategy that maximizes the overall Figure of Merit of a 3x2pt analysis. This setting is idealized as it ignores spectroscopic completeness  issues for the training sample, but complex enough to address the main question of binning optimization. Based on the challenge dataset, we review and compare the different methods that have been proposed by participants. We find that even from this limited photometry information, various methods are able to yield binning schemes that achieve high FoM scores.
\end{abstract}

\keywords{methods: statistical -- dark energy  -- large-scale structure of the universe}

%\accepted{}
%\submitjournal{the Astrophysical Journal Supplement}


%\tableofcontents%-----------------------------
%===========================
% BEGINNING OF THE MAIN TEXT
%===========================

\section{Introduction \& Motivation}

\begin{itemize}
    \item Lensing/clustering/3x2pt 
    \item Why bin galaxies for 3x2pt
    \item Current binning standards
    \item Constraints on potential binning methods
    \item Goal of challenge: identify binning method(s) leading to best cosmology constraints 
\end{itemize}

\url{https://arxiv.org/abs/1901.06495}

\url{https://arxiv.org/abs/astro-ph/0609338}

\section{Challenge Design}

\subsection{Philosophy}
\begin{itemize}
    \item Simplest challenge (optimistic data)
    \item Openness?
    \item Science-relevant metrics
    \item Performance at scale
\end{itemize}

\subsection{Data}

\begin{itemize}
    \item CosmoDC2 - description
    \item CosmoDC2 - discussion
    \item Noise properties
    \item Pre-selection 
    \item Metacal as a limitation
\end{itemize}

\subsection{Metrics}
\label{sec:metrics}


\begin{itemize}
    \item S/N of spectra, CCL
    \item DETF Figure of merit, FireCrown(?)
    \item mutual information
\end{itemize}


\placeholder{want to check performance at multiple stages of a cosmology analysis}

\subsection{Infrastructure}
\begin{itemize}
    \item Pull request submissions
    \item Data at NERSC
\end{itemize}


\subsection{Control Methods}
\begin{itemize}
    \item single bin
    \item one galaxy per bin?
    \item sanity check: random selection should be same as single bin
\end{itemize}


\section{Methods}
\begin{itemize}
    \item Gradient Boosting \@ jecampagne
    \item Neural Network \@EiffL
    \item Random Forest
    \item \placeholder{cuts in photometry space without redshift estimation \@barber?}
\end{itemize}

\section{Results}
\begin{itemize}
    \item Big table of results
    \item fisher matrix ellipses overplotted for top N methods
    \item Plots of n(z)
\end{itemize}

\section{Discussion}
\begin{itemize}
    \item Overall effectiveness of method classes
    \item Individual winner
    \item Largest number of bins that proves workable
\end{itemize}

\subsection{Future directions}

\begin{itemize}
    \item spectroscopic incompleteness
\end{itemize}

\section{Conclusions}



%=====================
% END OF THE MAIN TEXT
%=====================

\end{document}
